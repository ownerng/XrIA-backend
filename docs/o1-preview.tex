\documentclass[conference]{IEEEtran}
\IEEEoverridecommandlockouts
% The preceding line is only needed to identify funding in the first footnote. If that is unneeded, please comment it out.
\usepackage{cite}
\usepackage{amsmath,amssymb,amsfonts}
\usepackage{algorithmic}
\usepackage{graphicx}
\usepackage{textcomp}
\usepackage{xcolor}
\def\BibTeX{{\rm B\kern-.05em{\sc i\kern-.025em b}\kern-.08em
    T\kern-.1667em\lower.7ex\hbox{E}\kern-.125emX}}
\begin{document}

\title{Clasificador de radiografías de tórax para la detección de enfermedades\\

}

\author{\IEEEauthorblockN{Juan David Mateus}
\IEEEauthorblockA{\textit{Facultad de Ingeniería} \\
\textit{Universidad Cooperativa de Colombia}\\
Villavicencio, Colombia \\
juan.mateuss@campusucc.edu.co}
\and
\IEEEauthorblockN{Jhonnatan Mendez}
\IEEEauthorblockA{\textit{Facultad de Ingeniería} \\
\textit{Universidad Cooperativa de Colombia}\\
Villavicencio, Colombia \\
jhonnatan.mendez@campusucc.edu.co}
}

\maketitle

\begin{abstract}
La detección temprana y precisa de enfermedades pulmonares es un desafío crítico en la medicina moderna. Las radiografías de tórax son una herramienta clave para los diagnósticos clínicos, ya que permiten visualizar anormalidades en los pulmones y el corazón. Sin embargo, el análisis manual de estas imágenes por parte de radiólogos puede ser lento, costoso y propenso a errores humanos, especialmente cuando se revisan grandes volúmenes de radiografías en un entorno hospitalario con alta demanda.

La inteligencia artificial (IA) ha demostrado ser una tecnología prometedora en el campo de la medicina, con aplicaciones en el análisis de imágenes que ofrecen mayor rapidez y precisión en el diagnóstico. Este proyecto tiene como objetivo desarrollar un sistema automatizado para la clasificación de radiografías de tórax, mejorando así el rendimiento del diagnóstico médico y facilitando un tratamiento oportuno y adecuado para los pacientes.
\end{abstract}

\begin{IEEEkeywords}
Detección temprana, Enfermedades pulmonares, Radiografías de tórax, Inteligencia artificial (IA)
\end{IEEEkeywords}

\section{Introducción}
La detección temprana y precisa de enfermedades pulmonares es esencial para garantizar tratamientos efectivos y mejorar la calidad de vida de los pacientes. Las radiografías de tórax son una de las herramientas más utilizadas para el diagnóstico de afecciones pulmonares y cardíacas. No obstante, el proceso de análisis manual por parte de especialistas es propenso a errores y puede ser ineficiente en situaciones de alta demanda hospitalaria.

En este contexto, la inteligencia artificial (IA) emerge como una solución innovadora que puede asistir a los profesionales de la salud en el análisis de imágenes médicas. La implementación de sistemas automatizados de diagnóstico promete reducir el tiempo de análisis y aumentar la precisión en la detección de patologías, lo que resulta en beneficios significativos para el sistema de salud y los pacientes.

\section{Planteamiento del problema y/o justificación}
El diagnóstico tradicional de enfermedades pulmonares a través de radiografías de tórax depende en gran medida de la experiencia y habilidad del radiólogo. Esta dependencia puede conducir a inconsistencias en los diagnósticos, retrasos en la atención médica y, en algunos casos, a errores que afectan la salud del paciente.

Además, el creciente volumen de imágenes médicas que deben ser analizadas supera la capacidad humana, lo que genera la necesidad de herramientas que puedan automatizar y agilizar este proceso. La falta de soluciones tecnológicas avanzadas para la clasificación y detección de enfermedades en radiografías de tórax representa un obstáculo para la mejora de los servicios de salud.

La justificación de este proyecto radica en la necesidad de desarrollar un sistema que utilice IA para asistir en la detección de enfermedades pulmonares, mejorando la eficiencia y precisión del diagnóstico. Esto no solo aliviará la carga de trabajo de los radiólogos sino que también contribuirá a proporcionar un tratamiento más oportuno y adecuado a los pacientes.

\section{Objetivos}

\subsection{Objetivo general}
Desarrollar un sistema automatizado basado en inteligencia artificial para la clasificación de radiografías de tórax, que permita la detección temprana y precisa de enfermedades pulmonares, mejorando el diagnóstico médico y el tratamiento de los pacientes.

\subsection{Objetivos específicos}
\begin{itemize}
    \item Implementar un modelo de clasificación de imágenes utilizando técnicas de aprendizaje profundo y procesamiento de imágenes médicas.
    \item Integrar el modelo de clasificación en una aplicación web con frontend en React y backend en Django.
    \item Evaluar el rendimiento del sistema en términos de precisión, sensibilidad y especificidad en la detección de enfermedades pulmonares.
    \item Diseñar una interfaz de usuario intuitiva que facilite la interacción de los profesionales de la salud con el sistema.
\end{itemize}

\section{Marco referencial}

\subsection{Marco teórico y/o conceptual}
El proyecto se fundamenta en conceptos clave de inteligencia artificial, aprendizaje profundo y procesamiento de imágenes. Las redes neuronales convolucionales (CNN) son especialmente relevantes, ya que han demostrado un alto rendimiento en tareas de clasificación de imágenes médicas. Además, se considera la importancia de la integración de sistemas IA en entornos clínicos y los desafíos asociados, como la interpretación de resultados y la confianza en la tecnología.

En el ámbito del desarrollo web, se utilizan tecnologías modernas como React para el frontend y Django para el backend, permitiendo una arquitectura eficiente y escalable. Tailwind CSS se emplea para el diseño de la interfaz de usuario, asegurando una experiencia intuitiva y atractiva.

\section{Metodología}
La metodología del proyecto se divide en varias etapas:

\begin{enumerate}
    \item \textbf{Recolección y preprocesamiento de datos}: Obtención de un conjunto de datos de radiografías de tórax etiquetadas y aplicación de técnicas de preprocesamiento para mejorar la calidad de las imágenes.
    \item \textbf{Desarrollo del modelo de IA}: Implementación de una red neuronal convolucional utilizando frameworks como TensorFlow o PyTorch, y entrenamiento del modelo con los datos preprocesados.
    \item \textbf{Integración del modelo en el backend}: Desarrollo de una API RESTful en Django que permita interactuar con el modelo de IA y gestionar las solicitudes del frontend.
    \item \textbf{Desarrollo del frontend}: Creación de la interfaz de usuario con React y Tailwind CSS, facilitando la carga de imágenes y la visualización de resultados.
    \item \textbf{Pruebas y validación}: Evaluación del rendimiento del sistema mediante métricas como precisión, sensibilidad y especificidad, y realización de pruebas de usabilidad con usuarios finales.
    \item \textbf{Despliegue y documentación}: Implementación del sistema en un entorno de producción y elaboración de la documentación técnica y de usuario.
\end{enumerate}

\end{document}
