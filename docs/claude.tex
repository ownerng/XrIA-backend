\documentclass[conference]{IEEEtran}
\IEEEoverridecommandlockouts
\usepackage{cite}
\usepackage{amsmath,amssymb,amsfonts}
\usepackage{algorithmic}
\usepackage{graphicx}
\usepackage{textcomp}
\usepackage{xcolor}
\def\BibTeX{{\rm B\kern-.05em{\sc i\kern-.025em b}\kern-.08em
    T\kern-.1667em\lower.7ex\hbox{E}\kern-.125emX}}
\begin{document}

\title{Sistema de Detección de Enfermedades Pulmonares mediante Inteligencia Artificial aplicada a Radiografías de Tórax}

\author{\IEEEauthorblockN{Juan David Mateus}
\IEEEauthorblockA{\textit{Facultad de Ingenieria} \\
\textit{Universidad Cooperativa de Colombia}\\
Villavicencio, Colombia \\
juan.mateuss@campusucc.edu.co}
\and
\IEEEauthorblockN{Jhonnatan Mendez}
\IEEEauthorblockA{\textit{Facultad de ingenieria} \\
\textit{Universidad Cooperativa de Colombia}\\
Villavicencio, Colombia \\
jhonnatan.mendez@campusucc.edu.co}
}

\maketitle

\begin{abstract}
La detección temprana y precisa de enfermedades pulmonares representa un desafío significativo en la medicina moderna. Este proyecto presenta el desarrollo de un sistema basado en inteligencia artificial que automatiza la detección de patologías pulmonares a través del análisis de radiografías de tórax. El sistema integra tecnologías modernas como React con TypeScript y Tailwind para el frontend, y Django para el backend, creando una plataforma web robusta y accesible. La solución propuesta busca optimizar el proceso de diagnóstico médico, reduciendo el tiempo de análisis y mejorando la precisión en la detección de enfermedades pulmonares.
\end{abstract}

\begin{IEEEkeywords}
Inteligencia Artificial, Radiografías de Tórax, Diagnóstico Médico, React, Django, Aprendizaje Profundo
\end{IEEEkeywords}

\section{Introducción}
En la actualidad, el diagnóstico médico asistido por computadora ha emergido como una herramienta fundamental en el campo de la medicina moderna. La integración de tecnologías de inteligencia artificial con sistemas de procesamiento de imágenes médicas representa un avance significativo en la detección temprana y precisa de enfermedades pulmonares.

Este proyecto se centra en el desarrollo de una aplicación web que implementa un sistema de detección automatizada de enfermedades pulmonares mediante el análisis de radiografías de tórax. La solución combina tecnologías modernas de desarrollo web como React con TypeScript y Tailwind para la interfaz de usuario, y Django para la gestión del backend, creando así una plataforma integral que facilita el proceso de diagnóstico médico.

La implementación de este sistema busca abordar las limitaciones actuales en el análisis manual de radiografías, ofreciendo una herramienta que no solo acelera el proceso de diagnóstico sino que también mejora la precisión en la detección de patologías pulmonares.

\section{Planteamiento del Problema}
\subsection{Descripción del Problema}
El diagnóstico médico basado en radiografías de tórax enfrenta diversos desafíos en la actualidad:

\begin{itemize}
    \item Alta demanda de análisis radiológicos que resulta en tiempos de espera prolongados
    \item Variabilidad en la interpretación entre diferentes profesionales médicos
    \item Necesidad de procesamiento eficiente de grandes volúmenes de imágenes médicas
    \item Limitaciones en la disponibilidad de especialistas en áreas remotas
    \item Riesgo de error humano debido a fatiga o sobrecarga laboral
\end{itemize}

\subsection{Justificación}
El desarrollo de este sistema se justifica por múltiples factores:

\begin{itemize}
    \item Necesidad de automatizar y agilizar el proceso de diagnóstico médico
    \item Importancia de proporcionar herramientas de apoyo al personal médico
    \item Potencial para mejorar la accesibilidad a diagnósticos especializados
    \item Reducción de costos operativos en el sistema de salud
    \item Contribución al avance tecnológico en el campo de la medicina
\end{itemize}

\section{Objetivos}
\subsection{Objetivo General}
Desarrollar un sistema web basado en inteligencia artificial para la detección automatizada de enfermedades pulmonares mediante el análisis de radiografías de tórax, utilizando tecnologías modernas de desarrollo web y técnicas avanzadas de aprendizaje profundo.

\subsection{Objetivos Específicos}
\begin{itemize}
    \item Implementar una interfaz de usuario intuitiva utilizando React, TypeScript y Tailwind CSS que facilite la carga y visualización de radiografías
    \item Desarrollar un backend robusto en Django que gestione el procesamiento de imágenes y la implementación del modelo de IA
    \item Crear un modelo de aprendizaje profundo para la clasificación precisa de patologías pulmonares
    \item Establecer un sistema de almacenamiento y gestión eficiente de imágenes médicas
    \item Implementar medidas de seguridad y privacidad para la protección de datos médicos sensibles
\end{itemize}

\section{Marco Referencial}
\subsection{Marco Teórico}
El proyecto se fundamenta en diversos campos teóricos:

\begin{itemize}
    \item \textbf{Aprendizaje Profundo:} Utilización de redes neuronales convolucionales (CNN) para el análisis de imágenes médicas
    \item \textbf{Desarrollo Web Moderno:} Implementación de arquitecturas frontend-backend utilizando React y Django
    \item \textbf{Procesamiento de Imágenes Médicas:} Técnicas especializadas para el tratamiento de radiografías
    \item \textbf{Seguridad Informática:} Protocolos y estándares para la protección de datos médicos
\end{itemize}

\subsection{Metodología}
El desarrollo del proyecto sigue una metodología ágil adaptada:

\begin{enumerate}
    \item \textbf{Fase de Planificación}
    \begin{itemize}
        \item Análisis de requerimientos
        \item Definición de arquitectura del sistema
        \item Selección de tecnologías y herramientas
    \end{itemize}
    
    \item \textbf{Fase de Desarrollo}
    \begin{itemize}
        \item Implementación del frontend en React con TypeScript
        \item Desarrollo del backend en Django
        \item Entrenamiento del modelo de IA
    \end{itemize}
    
    \item \textbf{Fase de Pruebas}
    \begin{itemize}
        \item Validación del modelo de IA
        \item Pruebas de integración
        \item Evaluación de usabilidad
    \end{itemize}
    
    \item \textbf{Fase de Implementación}
    \begin{itemize}
        \item Despliegue del sistema
        \item Documentación
        \item Capacitación de usuarios
    \end{itemize}
\end{enumerate}

\end{document}