\documentclass[conference]{IEEEtran}
\IEEEoverridecommandlockouts
\usepackage{cite}
\usepackage{amsmath,amssymb,amsfonts}
\usepackage{algorithmic}
\usepackage{graphicx}
\usepackage{textcomp}
\usepackage{xcolor}
\def\BibTeX{{\rm B\kern-.05em{\sc i\kern-.025em b}\kern-.08em
    T\kern-.1667em\lower.7ex\hbox{E}\kern-.125emX}}
\begin{document}

\title{Clasificador de Radiografías de Tórax mediante Técnicas de Minería de Datos e Inteligencia Artificial\\}

\author{\IEEEauthorblockN{Juan David Mateus}
\IEEEauthorblockA{\textit{Facultad de Ingenieria} \\
\textit{Universidad Cooperativa de Colombia}\\
Villavicencio, Colombia \\
juan.mateuss@campusucc.edu.co}
\and
\IEEEauthorblockN{Jhonnatan Mendez}
\IEEEauthorblockA{\textit{Facultad de ingenieria} \\
\textit{Universidad Cooperativa de Colombia}\\
Villavicencio, Colombia \\
jhonnatan.mendez@campusucc.edu.co}
}

\maketitle

\begin{abstract}
La exploración y análisis de grandes volúmenes de datos médicos, específicamente radiografías de tórax, representa un desafío significativo en el campo de la minería de datos médicos. Este proyecto implementa técnicas avanzadas de minería de datos e inteligencia artificial para procesar y analizar radiografías de tórax, con el objetivo de detectar patrones y características asociadas a diversas enfermedades pulmonares. Utilizando técnicas de pre-procesamiento de imágenes, extracción de características y algoritmos de aprendizaje profundo, se desarrolla un sistema capaz de clasificar automáticamente diferentes patologías pulmonares, mejorando así la eficiencia y precisión del diagnóstico médico.
\end{abstract}

\begin{IEEEkeywords}
Minería de datos médicos, Procesamiento de imágenes, Deep Learning, Radiografías de tórax, Clasificación automática
\end{IEEEkeywords}

\section{Introduction}
En la era actual del Big Data médico, la minería de datos se ha convertido en una herramienta fundamental para el análisis y procesamiento de información médica. Las radiografías de tórax, como fuente rica en datos visuales, presentan un campo ideal para la aplicación de técnicas de minería de datos e inteligencia artificial. El proceso de extracción de conocimiento a partir de estas imágenes médicas involucra múltiples etapas, desde la recopilación y preprocesamiento de datos hasta la aplicación de algoritmos de aprendizaje profundo para la identificación de patrones patológicos.\\

Este proyecto implementa una metodología integral de minería de datos (KDD - Knowledge Discovery in Databases) aplicada al análisis de radiografías torácicas, combinando técnicas de procesamiento de imágenes, extracción de características y algoritmos de clasificación avanzados para desarrollar un sistema robusto de detección de enfermedades pulmonares.

\section{Planteamiento del problema}
\subsection{Descripcion del problema}
El análisis de grandes volúmenes de radiografías de tórax presenta desafíos significativos en el contexto de la minería de datos médicos:

\begin{itemize}
    \item Necesidad de procesar y analizar eficientemente grandes conjuntos de datos de imágenes médicas
    \item Complejidad en la extracción de características relevantes para el diagnóstico
    \item Variabilidad en la calidad y formato de las imágenes
    \item Requerimiento de técnicas avanzadas de preprocesamiento y normalización de datos
    \item Necesidad de algoritmos robustos para la clasificación automática de patologías
\end{itemize}

\subsection{Preguntas del problema}
\begin{enumerate}
    \item ¿Qué técnicas de minería de datos son más efectivas para el procesamiento y análisis de radiografías de tórax?
    \item ¿Cómo se puede optimizar el proceso de extracción de características relevantes en las imágenes médicas?
    \item ¿Qué algoritmos de clasificación proporcionan mejores resultados en la detección de patologías pulmonares?
    \item ¿Cómo se puede validar la efectividad del modelo de minería de datos en el contexto médico?
\end{enumerate}

\subsection{Objetivos}
\subsubsection{Objetivo general}
Desarrollar un sistema de clasificación automática de radiografías de tórax mediante técnicas de minería de datos y aprendizaje profundo, implementando una metodología KDD para la detección efectiva de enfermedades pulmonares.

\subsubsection{Objetivos especificos}
\begin{itemize}
    \item Implementar técnicas de preprocesamiento y normalización de imágenes médicas para optimizar la calidad de los datos
    \item Desarrollar algoritmos de extracción de características relevantes utilizando técnicas de minería de datos
    \item Diseñar e implementar un modelo de clasificación basado en redes neuronales convolucionales
    \item Evaluar y validar el rendimiento del sistema utilizando métricas estándar de minería de datos
\end{itemize}

\subsection{Justificacion}
La implementación de técnicas de minería de datos en el análisis de radiografías de tórax representa una solución innovadora a los desafíos actuales en el diagnóstico médico. Este enfoque permite:
\begin{itemize}
    \item Procesamiento eficiente de grandes volúmenes de datos médicos
    \item Extracción automática de características relevantes
    \item Mejora en la precisión del diagnóstico mediante algoritmos avanzados
    \item Reducción del tiempo de análisis y diagnóstico
    \item Apoyo objetivo en la toma de decisiones médicas
\end{itemize}

\section{Marco de referencia}
\subsection{Marco Teorico}
El proyecto se fundamenta en los siguientes conceptos y técnicas de minería de datos:

\begin{itemize}
    \item \textbf{Proceso KDD (Knowledge Discovery in Databases)}
    \begin{itemize}
        \item Selección y preprocesamiento de datos
        \item Transformación de datos
        \item Minería de datos
        \item Evaluación e interpretación de patrones
    \end{itemize}
    
    \item \textbf{Técnicas de Preprocesamiento de Imágenes}
    \begin{itemize}
        \item Normalización y estandarización
        \item Reducción de ruido
        \item Mejora de contraste
        \item Segmentación de regiones de interés
    \end{itemize}
    
    \item \textbf{Algoritmos de Aprendizaje Profundo}
    \begin{itemize}
        \item Redes neuronales convolucionales (CNN)
        \item Técnicas de transferencia de aprendizaje
        \item Optimización de hiperparámetros
    \end{itemize}
\end{itemize}

\section{Metodologia}
La metodología del proyecto se basa en el marco de trabajo Scrum, adaptado para proyectos de minería de datos:

\subsection{Metodología Scrum}
\begin{itemize}
    \item \textbf{Roles}
    \begin{itemize}
        \item Product Owner: Responsable de definir los requisitos del sistema
        \item Scrum Master: Facilitador del proceso
        \item Equipo de desarrollo: Implementación técnica
    \end{itemize}
    
    \item \textbf{Eventos}
    \begin{itemize}
        \item Sprint Planning: Planificación de tareas cada 2 semanas
        \item Daily Scrum: Reuniones diarias de seguimiento
        \item Sprint Review: Revisión de resultados
        \item Sprint Retrospective: Mejora continua del proceso
    \end{itemize}
\end{itemize}

\subsection{Fases del Proyecto}
\begin{enumerate}
    \item \textbf{Fase de Preparación de Datos}
    \begin{itemize}
        \item Recopilación de radiografías
        \item Preprocesamiento y normalización
        \item Etiquetado de datos
    \end{itemize}
    
    \item \textbf{Fase de Desarrollo del Modelo}
    \begin{itemize}
        \item Implementación de algoritmos de extracción de características
        \item Desarrollo del modelo de clasificación
        \item Optimización y ajuste de parámetros
    \end{itemize}
    
    \item \textbf{Fase de Evaluación}
    \begin{itemize}
        \item Validación del modelo
        \item Análisis de métricas de rendimiento
        \item Ajustes y mejoras iterativas
    \end{itemize}
    
    \item \textbf{Fase de Implementación}
    \begin{itemize}
        \item Desarrollo de la interfaz web
        \item Integración del modelo
        \item Pruebas de usuario
    \end{itemize}
\end{enumerate}

\end{document}